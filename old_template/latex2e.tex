\documentclass{article}  % Do NOT delete this line
\usepackage{lpiabstr_2e} % Do NOT delete this line
\usepackage{times}

% This is the macro file to use with LaTeX version 2e. It is written to
% work with the accompanying style file lpiabstr_2e.sty. 

% NOTE:  The style file lpiabstr.sty requires access to the Times 
% font package; if you do not have these fonts on your system, the 
% template and style file will not work. The font package can be 
% retrieved from any CTAN (Comprehensive TeX Archive Network) site;
% go to
%
% http://www.ctan.org/
%
% and click on "CTAN mirror" for a listing of the site nearest you.


\begin{document}  % Do NOT delete this line

% Type your title in the argument below, using upper- and lower-case letters:

\title{Global Evolution of Solids in Viscous Protoplanetary Disks}

% Type your authors as shown in the example below, listing each author in
% a separate author argument, with affiliation given in the affiliation 
% argument as needed:

\author{J. R. Dotson}
\affil{Lunar and Planetary Institute, Houston TX 77058-1113, USA,
(dotson@lpi.usra.edu)}
\author{D. Hilbert}
\author{S. L. Hokanson}
\author{L. Tanner}
\affil{Department of Astronomy, University of Houston-Clear Lake, 
Houston TX 77058, USA}
\author{A. Wiles}
\author{S. Lee}
\affil{Geology Department, University of Illinois, Chicago IL 60637, USA}

% If your abstract is two pages long, type your running head in the argument
% below:

\runningtitle{GLOBAL EVOLUTION: J. R. Dotson et al.}

\titlemake  % Sets the title, author information; do NOT delete

\begin{abstracttext}
This macro will format your abstract for the volume.
You will have retrieved this template file and the macro file lpiabstr\_2e.sty
from the meeting Web site.  The macro file lpiabstr\_2e.sty can be placed
in the same directory as your abstract, or in any of the standard LaTeX
input directories.  The macro file assumes that you are running LaTeX 2e;
if you are running LaTeX 2.09, you should use the 2.09 macros instead.

If your LaTeX system does not have a macro called ''times'' to configure
the use of PostScript Times Roman fonts, your system will complain that
it cannot find times.sty. Read the commented note found at the top of this
file for directions on where you can retrieve the Times Roman font package.

Insert the title of your abstract as the argument of the ``title'' macro,
just after the ``begin{document}'' macro.  Insert all author names and
affiliation names in the ``author'' and ``affil'' commands.  You should
list all the authors by institution, listing all the authors for the first
institution, then that institution, then all the authors for the second
institution, followed by the second institution, and so forth. If you
cannot group all the authors by affiliation, you will have to repeat
the affiliation for each separate occurrence. The example given in this
file shows the correct way to enter authors and affiliations (be sure to
remove all the samples when you prepare your abstract!). 

If the abstract is more than one page, you will need a running head for the
second page, abbreviating the title and author list as in the sample. 
Remember that the length of your abstract cannot exceed the length requirement
stated in the announcement text. If you
do not input the correct information for YOUR running head, the one used
in the sample will appear on the second page of your abstract.

You should then type your abstract between the command lines
``begin{abstracttext}'' and ``end{abstracttext}.'' Remember to remove
ALL the sample text that currently exists between these two lines.

After you run the LaTeX processor on your abstract, use dvips (or the dvi
processor on your system) to produce a PostScript output file.  You will
submit the PostScript file (and NOT the LaTeX file) to the LPI. (However,
depending on which meeting your abstract is for, the LPI may request your
source files separately.)

You will notice that macros are not given for internal formatting (headings,
etc.). The macros and style file merely define the page dimensions, format
of title and author lines, two-column text format, and running head
placement. Formatting of tables, equations, headings, captions, 
references, etc., are at your discretion. Keep in mind that your text
columns are only 3 inches (7.5 cm) wide, so plan the size of your tables, 
equations, and figures accordingly.

\end{abstracttext}

\end{document}

